\begin{abstract}
Basal melting of floating ice shelves fringing the Antarctic Ice Sheet (AIS) has been identified as a significant driver of uncertainty in sea level rise (SLR) projections. Part of this uncertainty derives from unpredictable internal variability in oceanic and atmospheric temperatures around ice shelves, in addition to poorly understood glaciological processes. However, quantifying this uncertainty with ensembles of climate and ice model simulations is prohibitively computationally expensive. Here, we develop and demonstrate a statistical technique that generates random realisations of basal melt rate projections which emulate a single long simulation from the Energy Exascale Earth System Model (E3SM) in a computationally efficient manner. This generation technique ensures spatially and temporally consistent variability, while also facilitating the sampling of a wide range of melt trajectories. Each such ensemble member can be characterised as an independent realisation of the ice sheet model simulation output with variable thermal forcing.
\end{abstract}