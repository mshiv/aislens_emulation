The manuscript should include a conclusion. In this section, summarize what was described in your paper. Future directions may also be included in this section. Authors are strongly encouraged not to reference multiple figures or tables in the conclusion; these should be referenced in the body of the paper. Looking forward statements. This technique can be used as a flexible source of ensemble members for future climate and ice sheet projections. We show that the projections of the basal melt arising from the generated alternate forcings are statistically consistent with those from the E3SM simulation in terms of the large-scale spatial variability. We also find that the method captures the variability in the timing of (peak?) melt rates. This generator can be applied to any large-scale ice sheet model that is forced by a global climate model and help to reduce the computational cost of generating large ensemble simulations of the AIS under historical and future forcing scenarios to better quantify SLR uncertainty arising from the AIS.